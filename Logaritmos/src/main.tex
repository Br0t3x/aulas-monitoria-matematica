\title{Exercícios de logaritmos}
\author{}
\date{}

\documentclass[a4paper, 11pt]{article}
\usepackage[utf8]{inputenc}
\usepackage[top=1cm, bottom=2cm, left=2cm, right=2cm]{geometry}
\usepackage{amsmath, amssymb, paralist, tabularx, enumitem}
\usepackage[brazil]{babel}
\usepackage{amsfonts}
\usepackage{booktabs}
\usepackage{amssymb}

\begin{document}
    \maketitle
    % Seção: Trabalhando com expoentes
    \begin{center}
        \section*{Trabalhando com expoentes}
    \end{center}

    % Lista de exercícios sobre expoentes
    \begin{enumerate}
        \item Simplifique as expressões, supondo $a\cdot b \neq 0$.

        \begin{tabularx}{\textwidth}{lXlXlXlX}
            a) & $(a^2 \cdot b^3)^2 \cdot (a^3 \cdot b^2)^3$ &
            b) & $\dfrac{(a^4 \cdot b^2)^3}{(a \cdot b^2)^2}$ \\[3ex]
            c) & $[(a^3\cdot b^2)^2]^3$ &
            d) & $\left(\dfrac{a^4\cdot b^3}{a^2\cdot b}\right)^5$ \\[3ex]
            e) & $\dfrac{(a^2\cdot b^3)^4\cdot(a^3\cdot b^4)^2}{(a^3\cdot b^2)^3}$ & & \\
        \end{tabularx} \\[2ex]
        \item Calcule o valor das expressões:

        \begin{tabularx}{\textwidth}{lXlXlX}
            a) & $\dfrac{2^{-1}-(-2)^2+(-2)^{-1}}{2^2-2^{-2}}$ &
            b) & $\dfrac{3^2-3^{-2}}{3^2+3^{-2}}$ &
            c) & $\dfrac{ \left(-\dfrac{1}{2} \right)^2 \cdot \left(\dfrac{1}{2} \right)^3} {\left[\left(-\dfrac{1}{2}\right)^2\right]^3} $\\
        \end{tabularx} \\[2ex]
        \item Remova os expoentes negativos e simplifique a expressão $\dfrac{x^{-1}+y^{-1}}{(xy)^{-1}}$, em que $x, y\in \mathbb{R}^*$.
        \item Se $a\cdot b \neq 0$, simplifique as expressões:\\

        \begin{tabularx}{\textwidth}{lXlXlXlX}
            a) & $(a^{-2}\cdot b^3)^{-2}\cdot(a^3\cdot b^{-2})^3$ &
            e) & $\dfrac{(a^3\cdot b^{-2})^{-2}\cdot(a\cdot b^{-2})^3}{(a^{-1}\cdot b^2)^{-3}}$\\[3ex]
            b) & $\dfrac{(a^5\cdot b^3)^2}{(a^{-4}\cdot b)^{-3}}$ &
            f) & $(a^{-1}+b^{-1})\cdot(a+b)^{-1}$ \\[3ex]
            c) & $[(a^2\cdot b^{-3})^2]^{-3}$ &
            g) & $(a^{-2}-b^{-2})\cdot(a^{-1}-b^{-1})^{-1}$ \\[2ex]
            d) & $\left(\dfrac{a^3\cdot b^{-4}}{a^{-2}\cdot b^2}\right)^3$ & & \\
        \end{tabularx}
    \end{enumerate}

    % Seção: Trabalhando com raízes
    \begin{center}
        \section*{Trabalhando com radicais}
    \end{center}

    % Lista de exercícios sobre raízes
    \begin{enumerate}
        \item sdikas
        \item daasd
    \end{enumerate}

    % Seção: trabalhando com logaritmos
    \begin{center}
        \section*{Trabalhando com logaritmos}
    \end{center}

    \begin{enumerate}
        \item Calcule pela definição os seguintes logaritmos: \\[1.5ex]
        \begin{tabularx}{\textwidth}{lXlXlX}
            \\
            a) & $\log_2{\dfrac{1}{8}}$ & b) & $\log_8{4}$ & c) & $\log_{0,25}{32}$ \\
        \end{tabularx}\\[2ex]

        \item Calcule pela definição os seguintes logaritmos: \\[1.5ex]
        \begin{tabularx}{\textwidth}{lXlXlX}
            a) & $$ & b) & $$ & c) & $$ \\

        \end{tabularx}\\[2ex]

        \item Calcule pela definição os seguintes logaritmos: \\[1.5ex]
        \begin{tabularx}{\textwidth}{lXlXlX}
            a) & $$ & b) & $$ & c) & $$ \\

        \end{tabularx}\\[2ex]

        \item Calcule o valor de: \\[1.5ex]
        \begin{tabularx}{\textwidth}{lXlXlX}
            a) & $$ & b) & $$ & c) & $$ \\

        \end{tabularx}\\[2ex]


        \item Calcule o valor de: \\[1.5ex]
        \begin{tabularx}{\textwidth}{lXlXlX}
            a) & $$ & b) & $$ & c) & $$ \\

        \end{tabularx}\\[2ex]

    \end{enumerate}

    % Seção: Desafios
    \begin{center}
        \section*{Desafios}
    \end{center}

    % Lista de desafios
    \begin{enumerate}
        \item Determine o menor número inteiro positivo $x$ para que $2940x = M^3$, em que $M$ é um número inteiro
        \item Qual o último algarismo do número $(14)^{(14)^{14}}$
    \end{enumerate}

    % Seção: Exercícios de vestibulares
    \begin{center}
        \section*{Exercícios de vestibulares}
    \end{center}

    % Lista de exercícios de vestibulares
    \begin{enumerate}
        \item (UFPB) A metade do número $2^{21}+4^{12}$ é:\\[1.5ex]
        \begin{tabularx}{\textwidth}{lXlXlXlXlX}
            a) & $2^{20}+2^{23}$ &
            b) & $2^{\frac{21}{2}}+4^6$ &
            c) & $2^{12}+4^{21}$ &
            d) & $2^{20}+4^6$ &
            e) & $2^{22}+4^{13}$ \\
        \end{tabularx}\\[2ex]

        \item (ENEM) A cor de uma estrela tem relação com a temperatura em sua superfície.
        Estrelas não muito quentes (cerca de 3.000 K) nos parecem avermelhadas. Já as estrelas amarelas, como o Sol,
        possuem temperatura em torno dos 6000 K; as mais quentes são brancas ou azuis porque sua temperatura
        fica acima dos 10000 K.\\
        A tabela apresenta uma classificação espectral e outros dados para as estrelas dessas classes.\\
        \begin{center}
            Estrelas da Sequência Principal \\
        \end{center}
        \begin{table}[h!]
            \centering
            \begin{tabular}{ccccc}
                \toprule
                \textbf{Classe espectral} & \textbf{Temperatura (K)} & \textbf{Luminosidade} & \textbf{Massa} & \textbf{Raio} \\
                \midrule
                05                        & 40.000                   & $5\cdot 10^5$         & 40             & 18            \\
                B0                        & 28.000                   & $2\cdot 10^4$         & 18             & 7             \\
                A0                        & 9.900                    & 80                    & 3              & 2,5           \\
                G2                        & 5.770                    & 1                     & 1              & 1             \\
                M0                        & 3.480                    & 0,06                  & 0,5            & 0,6           \\
                \bottomrule
            \end{tabular}\label{tab:table}
        \end{table}
        \caption{Luminosidade, massa e raio tomando o Sol como unidade}
        Se tomarmos uma estrela que tenha temperatura 5 vezes maior que a temperatura do Sol, qual será a ordem de grandeza de sua luminosidade? \\
        a) 20.000 vezes a luminosidade do Sol \\
        b) 28.000 vezes a luminosidade do Sol \\
        b) 28.850 vezes a luminosidade do Sol \\
        b) 30.000 vezes a luminosidade do Sol \\
        b) 50.000 vezes a luminosidade do Sol \\[1.5ex]
        \item (UFPR) Quando escrevemos 4.307, por exemplo, no sistema de numeração decimal,
        estamos nos referindo ao número $4\cdot10^3+3\cdot10^2+0\cdot10^1+7\cdot10^0$.
        Seguindo essa mesma ideia, podemos representar qualquer número inteiro positivo
        utilizando apenas os dígitos 1 e 0, bastando escrever o número como soma de potềncias
        de 2. Por exemplo, $13=1\cdot2^3+1\cdot2^2+0\cdot2^1+1\cdot2^0$ e por isso a notação
        $[1101]_2$ é usada para representar 13 nesse outro sistema. Note que os algarismos
        que ali aparecem são os coeficientes das potências de 2 na mesma ordem em que estão
        na expressão. Com base nessas informações, considere as seguintes afirmativas: \\
        I. $[111]_2=7$ \\
        II. $[110]_2+[101]_2=[1010]_2$ \\
        III. Independentemente do número inteiro positivo $k$, a expressão de $2^k$ em potências
        de 2 tem apenas um dígito diferente de 0. \\
        IV. Se $a=[\underbrace{111\ldots11}_{20\ \text{dígitos}}]_2,\text{então}\
        2\cdot a=[\underbrace{111\ldots110}_{21\ \text{dígitos}}]_2.$ \\[1.5ex]
        a) Somente as afirmativas I e III são verdadeiras. \\
        b) Somente as afirmativas II e III são verdadeiras. \\
        c) Somente as afirmativas I e IV são verdadeiras. \\
        d) Somente as afirmativas I, III e IV são verdadeiras. \\
        e) Somente as afirmativas II, III e IV são verdadeiras. \\[1.5ex]

        \item (PUC-PR) O valor de $x$ que satisfaz a equação $\dfrac{0,2^{x-0,5}}{\sqrt{5}}=5\cdot0,04^{x-1}$
        está compreendido no intervalo:
        \begin{tabularx}{\textwidth}{lXlXlXlXlX}
            a) & $x\leq0$ &
            b) & $0<x\leq1$ &
            c) & $1<x\leq4$ &
            d) & $4<x\leq20$ &
            e) & $x>20$\\
        \end{tabularx}\\[2ex]
        \item (EsPCEx-SP) A soma das raízes da equação $3^x+3^{1-x}=4$ é: \\[1.5ex]
        \begin{tabularx}{\textwidth}{lXlXlXlXlX}
            a) & 2 & b) & -2 & c) & 0 & d) & -1 & e) & 1 \\
        \end{tabularx} \\[2ex]
        \item (ITA-SP) Considere a equação $\dfrac{(a^x-a^{-x})}{(a^x+a^{-x})}=m$, na variável real $x$, com $0<a\neq1$.
        O conjunto de todos os valores de $m$ para os quais esta equação admite solução real é: \\[1.5ex]
        \begin{tabularx}{\textwidth}{lXlXlX}
            a) & $(-1,\ 0)\cup(0,\ 1)$ &
            c) & $(-1,\ 1)$ &
            e) & $(-\infty,\ +\infty)$ \\
            b) & $(-\infty,\ -1)\cup(1,\ +\infty)$ &
            d) & $(0,\ \infty)$ \\
        \end{tabularx} \\[2ex]
        \item (UFRR) Dados os conjuntos A = $\{x\in\mathbb{R}|9^{x^2}\leq243^{1-x}\}$ e B = $\{x\in\mathbb{R}|x^2+6x+9>0\}$,
        o conjunto A - B é igual a: \\[1.5ex]
        \begin{tabularx}{\textwidth}{lXlXlX}
            a) & $\emptyset$ & c) & $\left[-3,\ \dfrac{1}{2}\right]$ & e) & $\left{\dfrac{1}{2}\right}$ \\
            b) & ${-3}$ & d) & $]-\infty,\ -3[\ \cup\ ]-3,\ +\infty[$ \\
        \end{tabularx} \\[2ex]
        \item (PUC-PR) Sejam $x$ e $y$ dois números reais positivos tais que log x - log y = z, então
        log $\dfrac{1}{x}$ - log $\dfrac{1}{y}$ vale: \\[1.5ex]
        \begin{tabularx}{\textwidth}{lXlXlXlXlX}
            a) & $z$ &
            b) & $-z$ &
            c) & $z+1$ &
            d) & $-z+1$ &
            e) & 0\\
        \end{tabularx} \\[2ex]
        \item (Fuvest) Se $x$ é um número real, $x>2$ e $\log_2(x-2)-\log_4x=1$, então o valor de $x$ é:\\[1.5ex]
        \begin{tabularx}{\textwidth}{lXlXlXlXlX}
            a) & $4-2\sqrt3$ & b) & $4-\sqrt3$ & c) & $2+2\sqrt3$ & d) & $4+2\sqrt3$ & e) & $2+4\sqrt3$\\
        \end{tabularx} \\[2ex]
        \item (UFPR) Suponha que o tempo $t$ (em minutos) necessário para ferver água em um forno de micro-ondas
        seja dado pela função $t(n)=a\cdot n^b$, sendo $a$ e $b$ constantes e $n$ o número de copos de água que
        se deseja aquecer.
        \begin{center}
            \begin{table}[h!]
                \centering
                \begin{tabular}{cc}
                    \toprule
                    \textbf{Número de copos} & \textbf{Tempo de aquecimento} \\
                    \midrule
                    1                        & 1 minuto e 30 segundos        \\
                    2                        & 2 minutos                     \\
                    \bottomrule
                \end{tabular}\label{tab:table2}
            \end{table}\\[1.5ex]
        \end{center}
        a) Com base nos dados da tabela acima, determine os valores de $a$ e $b$. Sugestão: use log 2 = 0,3 e log 3 = 0,45.\\
        b) Qual é o tempo necessário para se ferverem 4 copos de água nesse forno de micro-ondas?\\
    \end{enumerate}
\end{document}